\section{Lab 2: Poisson's equation}

% One can thus implement the different schemes like red-black Gauss-Seidel iteration. This leads to
% a parallel program. However, there are numerous questions or problems that are worth to investigate.
% For instance, is it worth to make more than 1 iteration in each domain between two communication
% steps? The convergence of the algorithm will be slower if measured in the number of iteration steps,
% but if measured in real time it may or may not be advantageous. How much data should one transfer
% in a communication step? Just the data along the border, or maybe more layers at once? What is
% the best way to partition the grid points in to domains? Should these domains be as close to a square
% as possible, horizontal strips or vertical strips? How does the performance scale with problem size, or
% with increasing number of processes?

\subsection{Part 1}
\subsubsection{Step 1}

It is simple to understand that the program was indeed executed twice since two pairs of statements are written to the terminal in comparison to one pair before the modifications. Since we are now running the same program in two different nodes this behavior is expected.  

The result is the following
\begin{lstlisting}
	Number of iterations  : 2355
	Elapsed processortime : 1.350000 s
	Number of iterations  : 2355
	Elapsed processortime : 1.360000 s
\end{lstlisting}

\subsubsection{Step 2}

After adding the global variable and the necessary call to \texttt{MPI\_Comm\_rank} using the predefined communicator \texttt{MPI\_COMM\_WORLD}.
\begin{lstlisting}
	MPI_Comm_rank(MPI_COMM_WORLD, &proc_rank);
\end{lstlisting}

The following is printed to the standard output:
\begin{lstlisting}
	(0)     Elapsed processortime : 1.360000 s
	(0)     Number of iterations  : 2355
	(1)     Elapsed processortime : 1.360000 s
	(1)     Number of iterations  : 2355
\end{lstlisting}

\subsubsection{Step 3}

% After rewriting the timing functions mentioned in the exercise description, the new output is as follows:
% \begin{lstlisting}
%     (0)     Elapsed Wtime        : 1.410156 s ( 97.2% CPU)
%     (0)     Number of iterations : 2355
%     (1)     Elapsed Wtime        : 1.468750 s ( 94.0% CPU)
%     (1)     Number of iterations : 2355
% \end{lstlisting}

\subsubsection{Step 4}

Adjusting the code so each process writes to a separate file does not affect the text displayed, so there is no need to repeat it here. In addition by executing the command

\begin{lstlisting}
	diff output0.dat output1.dat
\end{lstlisting}

I was able to confirm the files are indeed identical.

\subsubsection{Step 5}

On this step, responsible to ensure correct distribution of information originated from an input file, several statements had to be rewritten. Below is a summary of those changes, in particular the parts that were not completely specified in the exercise manual.

To ensure only process 0 opens the file a simple comparison suffices
\begin{lstlisting}
	/* only process 0 may execute this if */
	if (proc_rank == 0)
	{ ... }
\end{lstlisting}

To broadcast the data read from the file it is first necessary to explain which fields the \texttt{MPI\_Bcast(void *buffer, int count, MPI\_Datatype datatype, int root, MPI\_Comm comm))} function requires. 

For our situation the \texttt{buffer} pointer should refer to the address of the variable we want to broadcast. The \texttt{count} relates to the number of entries in the buffer. The \texttt{datatype} should describe the type of data the buffer points to, \eg for integers this should be \texttt{MPI\_INT}. The \texttt{root} is the message broadcaster, in our case node 0. Finally we will use the usual predefined communicator for the last argument \texttt{comm}.

Thus the broadcast calls are as follows
\begin{lstlisting}
	/* broadcast the array gridsize in one call */
	MPI_Bcast(&gridsize      , 2, MPI_INT   , 0, MPI_COMM_WORLD);
	/* broadcast precision_goal */
	MPI_Bcast(&precision_goal, 1, MPI_DOUBLE, 0, MPI_COMM_WORLD);
	/* broadcast max_iter */  
	MPI_Bcast(&max_iter      , 1, MPI_INT   , 0, MPI_COMM_WORLD);
	(...)
	/* The return value of this scan is broadcast even though it is no input data */
	MPI_Bcast(&s, 1, MPI_INT, 0, MPI_COMM_WORLD);
	(...)
	/* broadcast source_x */
	MPI_Bcast(&source_x  , 1, MPI_DOUBLE, 0, MPI_COMM_WORLD);
	/* broadcast source_y */
	MPI_Bcast(&source_y  , 1, MPI_DOUBLE, 0, MPI_COMM_WORLD);
	/* broadcast source_val */
	MPI_Bcast(&source_val, 1, MPI_DOUBLE, 0, MPI_COMM_WORLD);
\end{lstlisting}

\subsubsection{Step 6}

Following the same approach as in the previous section, only the finished version of incomplete code from the manual will be shown in the excerpt.
\begin{lstlisting}
	MPI_Comm_size(MPI_COMM_WORLD, &P);
	(...)
	MPI_Cart_create(MPI_COMM_WORLD, 2, P_grid, wrap_around, reorder, &grid_comm);
	(...)
	/* Rank of process in new communicator */
	MPI_Comm_rank(grid_comm, &proc_rank);
	/* Coordinates of process in new communicator */
	MPI_Cart_coords(grid_comm, proc_rank, 2, proc_coord);
	(...)
	/* rank of processes proc_top and proc_bottom */
	MPI_Cart_shift(grid_comm, Y_DIR, 1, &proc_top, &proc_bottom);
	/* rank of processes proc_left and proc_right */
	MPI_Cart_shift(grid_comm, X_DIR, 1, &proc_left, &proc_right);
\end{lstlisting}

There a couple new function calls on this code whose arguments I will explain next.
As explained in the exercise description MPI uses the \texttt{MPI\_Cart\_*} function calls to arrange tasks in a virtual process grid.

To create one the API calls needs the previous communicator, in our case we were using \texttt{MPI\_COMM\_WORLD}. \texttt{ndims} and \texttt{dims} define the number of dimensions of the grid and the number of processors in each, respectively. For our example these are 2 and \texttt{P\_grid}. Thereafter the periods specifies if the grid is periodic or not per dimension, and finally if the ranking is reordered or not. These are replaced by the self-explanatory variables \texttt{wrap\_around} and \texttt{reorder}. The new communicator is stored in the address of \texttt{comm\_cart}. From now on all pointers to communicators refer to this one.

\begin{lstlisting}
	int MPI_Cart_create(MPI_Comm comm_old, int ndims, int *dims, int *periods, int reorder, MPI_Comm *comm_cart)
	int MPI_Cart_coords(MPI_Comm comm, int rank, int maxdims, int *coords)
	int MPI_Cart_shift(MPI_Comm comm, int direction, int displ, int *source, int *dest)
\end{lstlisting}

To define the coordinates of the process in the new communicator we use \texttt{MPI\_Cart\_coords}. Here the rank is necessarily the processor rank (\texttt{proc\_rank}), the maximum dimensions is 2. The coordinates of each specified process are stored in the \texttt{proc\_coord} array.

Finally the shift operation returns the shifted source and destination ranks. The direction and displacement quantity arguments are self-evident and are replaced in the program by \texttt{X\_DIR} or \texttt{Y\_DIR} and 1 respectively for horizontal and vertical displacements. In accordance to the outputs source and destination are stored in \texttt{proc\_top} and \texttt{proc\_bottom} or \texttt{proc\_left} and \texttt{proc\_right} depending on the direction. 


The text written to the standard output now features the coordinate of each processor.

\begin{lstlisting}
	(0) Number of iterations : 2355
	(0) (x,y)=(0,0)
	(0) Elapsed Wtime:       1.414062 s ( 96.9% CPU)
	(1) Number of iterations : 2355
	(1) (x,y)=(1,0)
	(1) Elapsed Wtime:       1.464844 s ( 95.6% CPU)
\end{lstlisting}

As one can observe since there are two processor, number 0 is allocated the left half of the grid, and processor with rank 1 deals with the right half.

\subsubsection{Step 7}
% After adjusting the \texttt{Setup\_Grid()} function as described the following results were obtained for three different setups. The first has two processors, the second three and the third four. The results can be seen below.

% \begin{lstlisting}
    % (0)   Number of iterations : 1195
    % (0) (x,y)=(0,0)
    % (0) Elapsed Wtime:       0.394531 s ( 86.2% CPU)
    % (1)   Number of iterations : 945
    % (1) (x,y)=(1,0)
    % (1) Elapsed Wtime:       0.347656 s ( 83.4% CPU)

    % (0)   Number of iterations : 1
    % (0) (x,y)=(0,0)
    % (0) Elapsed Wtime:       0.007812 s (  0.0% CPU)
    % (1)   Number of iterations : 695
    % (1) (x,y)=(1,0)
    % (1) Elapsed Wtime:       0.214844 s ( 79.1% CPU)
    % (2)   Number of iterations : 1
    % (2) (x,y)=(2,0)
    % (2) Elapsed Wtime:       0.046875 s (106.7% CPU)

    % (0)   Number of iterations : 791
    % (0) (x,y)=(0,0)
    % (0) Elapsed Wtime:       0.128906 s ( 85.3% CPU)
    % (1)   Number of iterations : 792
    % (1) (x,y)=(0,1)
    % (1) Elapsed Wtime:       0.160156 s ( 99.9% CPU)
    % (2)   Number of iterations : 1
    % (2) (x,y)=(1,0)
    % (2) Elapsed Wtime:       0.058594 s ( 68.3% CPU)
    % (3)   Number of iterations : 791
    % (3) (x,y)=(1,1)
    % (3) Elapsed Wtime:       0.164062 s ( 97.5% CPU)
% \end{lstlisting}

When there are three processors the work can not be evenly split between them. This can be confirmed by inspecting the \texttt{x} and \texttt{y} variables in the \texttt{Setup\_grid function}.

For example for processor 2 (in a 3 processor configuration) it is visible that \texttt{x} is always negative.

\begin{lstlisting}
	(2) x = -30, dim[X_DIR] = 36
	(2) y =  71,  dim[Y_DIR] = 102

	(2) x = -3, dim[X_DIR] = 36
	(2) y = 76, dim[Y_DIR] = 102

	(2) x = -28, dim[X_DIR] = 36
	(2) y =  26, dim[Y_DIR] = 102
\end{lstlisting}


\subsubsection{Step 8}
\subsubsection{Step 9}

After implementing the collective reduction operation the total number of iterations is indeed the same, as confirmed by the program's output.

\begin{lstlisting}
	(0) Number of iterations : 2355
	(0) Elapsed Wtime:       1.507812 s ( 96.2% CPU)
	(1) Number of iterations : 2355
	(1) Elapsed Wtime:       1.496094 s ( 97.6% CPU)
	(2) Number of iterations : 2355
	(2) Elapsed Wtime:       1.507812 s ( 96.8% CPU)
\end{lstlisting}

\subsubsection{Step 10}


\subsection{Part 2}

\subsubsection{2.1}
\subsubsection{2.2}

After changing the code to accommodate for the algorithmic improvement, 5 tests were performed to compare different values for the relaxation parameter $\omega$. The results are summarized in table~\ref{tbl:omega}.
From this data we concluded that 1.93 is the optimal value for the relaxation parameter, accomplishing almost 18 times less iterations than the original with $\omega$ equal to one.

\begin{table}[H]
\centering
\begin{tabular}{cccc}
 \toprule
$\omega$ & \texttt{Wtime}\tablefootnote{This value was computed as the maximum \texttt{Wtime} over the four individual processor times for each $\omega$ value.} (\si{s}) & $n$ & Reduction\\ \midrule
    1.00 &        1.500000 &        2355 &        1.00 \\    
    1.90 &        0.222656 &        220  &        10.7 \\
    1.92 &        0.199219 &        165  &        14.3 \\
\bf 1.93 &    \bf 0.175781 &    \bf 131  &    \bf 18.0 \\
    1.94 &        0.289062 &        142  &        16.6 \\
    1.98 &        0.351562 &        419  &        5.62 \\
\bottomrule
\end{tabular}
\caption{Time, number of iterations obtained and respective iteration reduction for different $\omega$ values. The topology used was \texttt{pt:}441 with a grid size of \texttt{g:}100x100.}
\label{tbl:omega}
\end{table}


\subsubsection{2.3}

The goal of this exercise is to investigate the scaling behavior of the code with a fixed relaxation parameter. To accomplish that analysis several runs were measured with various grid sizes. In addition different \emph{slices} were also tested.

\begin{table}[H]
\centering
\begin{tabular}{*{5}{c}}
 \toprule
          &     & \multicolumn{2}{c}{\texttt{Wtime} (\si{s})} & \\
Grid Size & $n$ & \texttt{pt:}441 & \texttt{pt:}422           & $\delta$ (\%)\\ \midrule
\multirow{4}{*}{200}
    &    50   &     0.164 &     0.164 &     0.00 \\
    &    100  &     0.226 &     0.246 &     8.62 \\
    &    200  &     0.343 &     0.335 &    -2.27 \\
    &    300  &     0.464 &     0.449 &    -3.36 \\\\
\multirow{4}{*}{400}                                        
    &    100  &     0.437 &     0.460 &     5.36 \\
    &    300  &     1.054 &     1.054 &     0.00 \\
    &    500  &     1.746 &     1.667 &    -4.47 \\
    &    1000 &     3.308 &     3.304 &    -0.12 \\\\
\multirow{4}{*}{800}
    &    100  &     1.414 &     1.445 &     2.21 \\
    &    300  &     3.742 &     3.726 &    -0.42 \\
    &    500  &     5.976 &     5.988 &     0.20 \\
    &    1000 &    11.625 &    11.617 &    -0.07 \\\\
\multirow{4}{*}{2000}
    &    100  &     8.218 &     8.234 &     0.19 \\
    &    300  &    21.351 &    21.328 &    -0.11 \\
    &    500  &    34.933 &    35.492 &     1.60 \\
    &    1000 &    68.304 &    67.824 &    -0.70 \\
\bottomrule
\end{tabular}
\caption{The maximum time for different grid sizes and different \emph{slicing} arrangements. The $\delta$ column compares the relative difference in performance between the two previous columns. The number of iterations for each run was fixed to enable a more accurate comparison. The 200x200 grid size measurements were performed in different conditions in regards to the iteration count since the program converges after 382 iterations.}
\label{tbl:slices}
\end{table}

The results of the aforementioned experiment are shown in table~\ref{tbl:slices}. As one can obverse the different domain partitions have little impact in the overall performance of the program, even as the grid size increases.


As explained in the exercise manual the (average) time $t$ per iteration $n$ can be parametrized as follows
\begin{equation}
t(n) = \alpha + \beta \cdot n
\label{eq:alphabeta}
\end{equation}

Where $\alpha$ and $\beta$ are arbitrary constants. To determine these constants we will use the least-squares method. The results of applying this technique\tablefootnote{The Google Drive implementation of least-squares, \emph{LINEST} function, was used for this computation. For more information please see \url{https://support.google.com/drive/answer/3094249}.} to the available dataset are shown in table~\ref{tbl:alphabeta}

\begin{table}[H]
\centering
\begin{tabular}{*{5}{c}}
 \toprule
            &    \multicolumn{2}{c}{\texttt{pt:}441} & \multicolumn{2}{c}{\texttt{pt:}422} \\
Grid Size   &    $\alpha$ &    $\beta$                & $\alpha$    & $\beta$ \\ \midrule
200         &    0.105    &    0.001                  & 0.120       & 0.001   \\
400         &    0.116    &    0.003                  & 0.115       & 0.003   \\
800         &    0.310    &    0.011                  & 0.329       & 0.011   \\
2000        &    1.449    &    0.067                  & 1.721       & 0.066   \\
\bottomrule
\end{tabular}
\caption{Using least-squares method we estimated $\alpha$ and $\beta$ as defined in equation~\ref{eq:alphabeta}. The dataset adopted for this computation is present in table~\ref{tbl:slices}.}
\label{tbl:alphabeta}
\end{table}

In order to improve the quality of this an estimation a thorougher study should be performed with more runs. In addition considering other (non-linear) parametrization functions, \eg exponential, may also yield interesting results.

To conclude this question we plotted the estimated functions along with the actual data points to facilitate a visual comparison of the estimation.


\begin{figure}[ht]
\centering
\begin{tikzpicture}[scale=1]
\begin{groupplot}[
    group style={
        group name=my fancy plots,
        group size=1 by 3,
        xticklabels at=edge bottom,
        vertical sep=0pt
    },
    width=0.9\textwidth,
    xmin=0, xmax=13,
]

\nextgroupplot[ymin=5.35,ymax=90,
               ytick={5, 25, 45, 65},
               axis line style = ultra thin,
               axis x line=none,
               axis y line=left,
               % axis y discontinuity=crunch,
               ylabel=$t(n)$,
               ylabel style={at={(-0.02,0.05))},rotate=270},
               height=6.0cm,
               mark size = 3,
               very thick]
    \addplot[color=green, domain=0:12]           		{0.011324*100*x+0.310667} ; %node[above right] {441-800x800};
    \addplot[color=green!50!black, domain=0:12,dotted]  {0.01130*100*x+0.32913}   ; %node[below right] {422-800x800};
    \addplot[color=orange, domain=0:12]          		{0.066849*100*x+1.449066} ; %node[below left] {441-2000x2000};
    \addplot[color=orange!70!black, domain=0:12,dotted] {0.06631*100*x+1.72143}   ; %node[below left] {422-2000x2000};
    
    \addplot[color=green, only marks, mark=square*, ]           file {441-800x800.data};
    \addplot[color=green!50!black, only marks, mark=triangle*]  file {422-800x800.data};
    \addplot[color=orange, only marks, mark=square*]            file {441-2000x2000.data};
    \addplot[color=orange!70!black, only marks, mark=triangle*] file {422-2000x2000.data};


\nextgroupplot[ymin=4,ymax=5,
               height=2.5cm,
               ymajorticks=false,
               axis line style = ultra thin,
               axis x line=none,
               axis y line=left,
               axis y discontinuity=crunch,
               y axis line style=-,
               very thick]
    \addplot[color=green, domain=0:12]           		{0.011324*100*x+0.310667} ; %node[above right] {441-800x800};
    \addplot[color=green!50!black, domain=0:12,dotted]  {0.01130*100*x+0.32913}   ; %node[below right] {422-800x800};
    \addplot[color=orange, domain=0:12]          		{0.066849*100*x+1.449066} ; %node[below left] {441-2000x2000};
    \addplot[color=orange!70!black, domain=0:12,dotted] {0.06631*100*x+1.72143}   ; %node[below left] {422-2000x2000};
    
    \addplot[color=green, only marks, mark=square*, ]           file {441-800x800.data};
    \addplot[color=green!50!black, only marks, mark=triangle*]  file {422-800x800.data};
    \addplot[color=orange, only marks, mark=square*]            file {441-2000x2000.data};
    \addplot[color=orange!70!black, only marks, mark=triangle*] file {422-2000x2000.data};


\nextgroupplot[ymin=0,ymax=4,
               ytick={0, 1, 2, 3},
               xtick={1, 2, 3, 5, 10},
               axis line style = ultra thin,
               axis x line=bottom,
               axis y line=left,
               y axis line style=-,
               xlabel = $n/100$,
               height=6.0cm,
               mark size = 3,
               very thick]
    \addplot[color=green, domain=0:12]           		{0.011324*100*x+0.310667} ; %node[above right] {441-800x800};
    \addplot[color=green!50!black, domain=0:12,dotted]  {0.01130*100*x+0.32913}   ; %node[below right] {422-800x800};
    \addplot[color=orange, domain=0:12]          		{0.066849*100*x+1.449066} ; %node[below left] {441-2000x2000};
    \addplot[color=orange!70!black, domain=0:12,dotted] {0.06631*100*x+1.72143}   ; %node[below left] {422-2000x2000};
    
    \addplot[color=green, only marks, mark=square*, ]           file {441-800x800.data};
    \addplot[color=green!50!black, only marks, mark=triangle*]  file {422-800x800.data};
    \addplot[color=orange, only marks, mark=square*]            file {441-2000x2000.data};
    \addplot[color=orange!70!black, only marks, mark=triangle*] file {422-2000x2000.data};


    \addplot[color=red, domain=0:12]             		{0.001198*100*x+0.105071}  ;% node[above right] {441-200x200};
    \addplot[color=red!50!black, domain=0:12,dotted]    {0.00110*100*x+0.12023}    ;% node[below right] {422-200x200};
    \addplot[color=blue, domain=0:12]            		{0.003200*100*x+0.116686}  ;% node[above right] {441-400x400};
    \addplot[color=blue!50!white, domain=0:12,dotted]   {0.00317*100*x+0.11562}    ;% node[below right] {422-400x400};
    
    \addplot[color=red, only marks, mark=square*]             file {441-200x200.data};
    \addplot[color=red!50!black, only marks, mark=triangle*]  file {422-200x200.data};
    \addplot[color=blue, only marks, mark=square*]            file {441-400x400.data};
    \addplot[color=blue!50!white, only marks, mark=triangle*] file {422-400x400.data};



\end{groupplot}
    
    % legend
    \begin{scope}[shift={(0, 0)}, mark size = 3] 

    \draw[yshift=3\baselineskip]                                  (5.5, 4) -- plot[mark=triangle*, mark options={fill=white}] (6.0, 4) -- (6.5, 4) node[right]{\texttt{pt:}422};
    \draw[yshift=3\baselineskip] (0, 4) node[right]{data points:} (2.5, 4) -- plot[mark=square*, mark options={fill=white}]   (3.0, 4) -- (3.5, 4) node[right]{\texttt{pt:}441};
    
    \draw[ultra thick, yshift=2\baselineskip, dotted]                        (5.5, 4) -- (6.5, 4) node[right]{\texttt{pt:}422};
    \draw[ultra thick, yshift=2\baselineskip] (0,4) node[right]{estimation:} (2.5, 4) -- (3.5, 4) node[right]{\texttt{pt:}441};

    \draw[ultra thick, yshift=1\baselineskip, draw=blue]                  (5.5,4) -- (6.0,4) ;
    \draw[ultra thick, yshift=1\baselineskip, draw=blue!50!white, dotted] (6.0,4) -- (6.5,4) node[right]{\texttt{g:}400};

    \draw[ultra thick, yshift=1\baselineskip, draw=red]                   (2.5, 4) -- (3.0, 4) ;
    \draw[ultra thick, yshift=1\baselineskip, draw=red!50!black, dotted]  (3.0, 4) -- (3.5, 4) node[right]{\texttt{g:}200};
    
    \draw[ultra thick, draw=orange]                                       (5.5,4) -- (6.0,4) ;
    \draw[ultra thick, draw=orange!70!black, dotted]                      (6.0,4) -- (6.5,4) node[right]{\texttt{g:}2000};
    
    \draw[ultra thick, draw=green]                                        (2.5, 4) -- (3.0, 4) ;
    \draw[ultra thick, draw=green!50!black, dotted]                       (3.0, 4) -- (3.5, 4) node[right]{\texttt{g:}800};
    \end{scope}

\end{tikzpicture}
\caption{Visual comparison between the experimental data and respective linear estimation using the least-squares method.}
\label{fig:plot}
\end{figure}

As figure~\ref{fig:plot} shows the estimated $\alpha$ and $\beta$ can accurately determine the time necessary to complete the operation. This can be observed by noticing that the measured data points (depicted as
\tikz{\draw (0,0) -- plot[mark size = 3, mark=triangle*, mark options={fill=white}] (0.25,0) -- (0.5,0);}
and
\tikz{\draw (0,0) -- plot[mark size = 3, mark=square*, mark options={fill=white}] (0.25,0) -- (0.5,0);}
for the two different partitions)
are located close to the curves (which represent the predicted values).


\subsubsection{2.4}

Based on the results from the previous question it is expected that the division of the domains does not result is significantly different running times. Thus, the choice can be done arbitrarily. In order to estimate the constants without performing further experiments one can extrapolate based on the already calculated values.



For 8 processors the area allocated to each processor is reduced 4 times in comparison with 4 processors. Therefore it is reasonable to assume that the time per iteration ($\beta$)is reduced by the same ratio. On the other the communication overhead ($\alpha$) is not expected to experience drastic changes. After identifying this assumptions it is trivial to estimate the new constants.

\begin{eqnarray}
\alpha_8 &=& \alpha_4 \label{eq:newalpha}\\
\beta_8  &=& \beta_4 / 4 \label{eq:newbeta}
\end{eqnarray}

Table~\ref{tbl:newalphabeta} incorporates equations~\ref{eq:newalpha} and~\ref{eq:newbeta} to predict the new constants.

\begin{table}[H]
\centering
\begin{tabular}{*{5}{c}}
 \toprule
            &    \multicolumn{2}{c}{\texttt{pt:}441} & \multicolumn{2}{c}{\texttt{pt:}422} \\
Grid Size   &    $\alpha_8$ &    $\beta_8$           & $\alpha_8$    & $\beta_8$ \\ \midrule
200         &    0.105      &    0.00029             & 0.120         & 0.00027   \\
400         &    0.116      &    0.00080             & 0.115         & 0.00079   \\
800         &    0.310      &    0.00283             & 0.329         & 0.00282   \\
2000        &    1.449      &    0.01671             & 1.721         & 0.01657   \\
\bottomrule
\end{tabular}
\caption{The estimated $\alpha$ and $\beta$ constants for an 8 processor configuration. These results are based on the data from table~\ref{tbl:alphabeta} and equations~\ref{eq:newalpha} and \ref{eq:newbeta}.}
\label{tbl:newalphabeta}
\end{table}

\subsubsection{2.5}

Table~\ref{tbl:iterations} features the number of iterations necessary for several grid sizes.

\begin{table}[H]
\centering
\begin{tabular}{*{2}{c}}
 \toprule
$g$   &    $n$   \\ \midrule
200   &    382   \\
300   &    771   \\
400   &   1206   \\
500   &   1664   \\
\bottomrule
\end{tabular}
\caption{Number of iterations necessary to solve the Poisson equation for various grid sizes. The topology used was \texttt{pt:}441.}
\label{tbl:iterations}
\end{table}

Using the same technique as before we can parametrize the iteration evolution according to grid size to a linear equation.
For the aforementioned data we obtained the following constants $\alpha = -492.6$ and $\beta = 4.281$.
Thus equation~\ref{eq:iterations} can be used to estimate the number iterations for different grid sizes.

\begin{eqnarray}
n(g) = \alpha + g \cdot \beta = -492.6 + 4.281 \cdot g\\
\label{eq:iterations}
\end{eqnarray}

To finish table~\ref{tbl:est-iterations} highlights the errors between the predictions and the measured values, and estimates the number of iterations for higher grid sizes.

\begin{table}[H]
\centering
\begin{tabular}{*{4}{c}}
 \toprule
$g$      &    $n$   &   $n_{\text{est}}$   &   $\delta$ (\%) \\ \midrule
200      &    382   &   364                &    4.82         \\
300      &    771   &   792                &   -2.68         \\
400      &   1206   &   1220               &   -1.14         \\
500      &   1664   &   1648               &    0.97         \\\\
1000     &    -     &   3788               &     -           \\
5000     &    -     &   20912              &     -           \\
10000    &    -     &   42317              &     -           \\
\bottomrule
\end{tabular}
\caption{Comparing empirical and estimated data for iteration evolution with increasing grid sizes.}
\label{tbl:est-iterations}
\end{table}

The conclusion that can be drawn from table~\ref{tbl:est-iterations} is that the iteration evolution is approximately linear with the grid size. This claim is supported by the low relative error between predicted and actual data points, in particular as dimensions increase. Nonetheless in order to make a stronger claim it is important to perform a more refined study (\ie increase the data set) as we only consider four point.


\subsubsection{2.6}

\begin{figure}[ht]
\centering
\begin{tikzpicture}

    % \begin{axis}[
    \begin{semilogyaxis}[
        xlabel=$n$,
        ylabel={$\delta_{\text{global}}$},
        ylabel style={at={(-.1,0.5)}},
        % restrict y to domain=0:1,
        % restrict x to domain=0:100
        ymin=0.00005,ymax=20,
        xmin=0,xmax=2000,
        xtick={0, 400, 800, 1200, 1600},
        ytick={0.0001, 0.001, 0.01, 0.1, 1.0},
        axis line style = ultra thin,
        axis x line=left,
        axis y line=left,
        colormap={redblue}{color=(red) color=(blue)},
        % axis y discontinuity=crunch,
    ]
    \addplot[scatter,only marks, mark size=0.9, each nth point={1}, filter discard warning=false, scatter src=y, ultra thin] file {data/global-delta.data};
    \end{semilogyaxis}
\end{tikzpicture}
\caption{The evolution of $\delta_{\text{global}}$ for a grid size of 500 and a 41 processor topology. The blue color represents high error while red illustrates errors close to the precision goal, \num{e-4}.
When this value is reached no more iterations occur. It is important to note that the $y$ axis is logarithmic.}
\label{fig:error}
\end{figure}

Figure~\ref{fig:error} depicts the evolution of the error with increased iterations for a 500x500 grid size. It is worthy to mention that the error reduces drastically until approximately the hundredth iteration. From that point on the rate of descent is decreased


\subsubsection{2.8}

Changing the number of red/black sweeps between border exchanges has a significant impact both on the total number of iterations and the total running time, as exhibited in table~\ref{tbl:sweeps}. As the work between communication steps increases the number of iterations to converge declines, however the time per iteration augments. The issue, is that the latter rate is higher than the former which results in a global execution time penalty.

\begin{table}[H]
\centering
\begin{tabular}{l*{2}{c}}
 \toprule
Sweeps  &   $n$   &   \texttt{Wtime} (\si{s}) \\ \midrule
1       &   1664  &   7.945312 \\
2       &   927   &   8.125000 \\
3       &   732   &   9.191406 \\
5       &   624   &  12.808594 \\
\bottomrule
\end{tabular}
\caption{Analyzing the execution time as the number of sweeps between communication steps evolves. $n$ portrays the number of iterations necessary for convergence. These results were obtained with the following configuration \texttt{g:}500x500 and \texttt{pt:}441.}
\label{tbl:sweeps}
\end{table}


\subsubsection{2.9}

To incorporate the optimization indicated in the question an extra variable, \texttt{aux}, was added.
The respective code, in particular the inner \texttt{for} loop was updated to the following.

\begin{lstlisting}
for (x = 1; x < dim[X_DIR] - 1; x++) {
	aux = (x + offset[X_DIR] + offset[Y_DIR] + parity + 1) % 2;
	for (y = 1 + aux; y < dim[Y_DIR] - 1; y += 2)  
		if (source[x][y] != 1)
		{
			...
		}
	}
}
\end{lstlisting}

The added variable along with the adjustment to the increment of \texttt{y} obviates the need to explicitly check the parity in the subsequent \texttt{if} statement.
More importantly it reduces the number of iterations of the inner loop. Table~\ref{tbl:optimization} compares the running times foe before and after the optimization.


\begin{table}[H] %TODO
\centering
\begin{tabular}{*{4}{c}}
 \toprule
\texttt{gs}  &   \texttt{Wtime\textsubscript{before}} (\si{s})  & \texttt{Wtime\textsubscript{after}} (\si{s})  & $\delta$ (\%) \\ \midrule
200          &   9.999999                                       & 9.999999                                      &  9.9999       \\
300          &   9.999999                                       & 9.999999                                      &  9.9999       \\
500          &   9.999999                                       & 9.999999                                      &  9.9999       \\
\bottomrule
\end{tabular}
\caption{Results before and after implementing the inner loop optimization for various grid sizes.}
\label{tbl:optimization}
\end{table}

\subsubsection{2.11}


\subsubsection{2.12}